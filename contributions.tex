\section{Contributions}\label{sec:contrib}

On the basis of the qualitative analysis of the problems and of the main aspects related to jumping with robots which is outlined in Sec.~\ref{sec:prb_def}
The pipeline is made of an online and an offline layers (see Fig.~\ref{fig:pipeline}). The online layer is occupied by a simple yet-effective joint level impedance controller, while the offline layer is made of two separate TO problems. The first one addresses the take-off and flight phases, while the second one tackles the impact and braking phases. Our choice of grouping together the first and last two phases of a jump stems very naturally from the considerations highlighted in Sec.~\ref{sec:prb_def}. Furthermore, we choose to exploiting Trajectory Optimization (TO), which has become one of the most successful methods for generating complex motions of robotic systems\cite{agile_bots::neunert2017trajectory,agile_bots::winkler2018gait,agile_bots::chignoli2021online,agile_bots::nguyen2019optimized,agile_bots::chignoli2021humanoid, agile_bots::roscia2023orientation,agile_bots::carius2019trajectory} in recent years, due to its potential in accounting for the performance of the hardware. The benefits of TO are well-known: stability, robustness and interpretability.