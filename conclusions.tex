\section{Conclusions, challenges and future work}\label{sec:conclusions}
In this work we presented a case study built around the generation and execution of agile jumping maneuvers on a quadruped leg prototype.  
% In order to fully exploit and asses the performance of our single leg prototype, we employed accurate models of both the actuators and of the energy flow over the power bus. 
%Empowered with accurate actuation and energy flow models,
We explored the peculiar challenges associated to each of the phases of the jumping motion and proposed a two-stage solution to the generation of agile jumping maneuvers which relies on two sequential TO problems: one for addressing the thrust phase with the objective of maximizing the jump height and the other to address the braking phase, with the objective of maximizing the energy regenerated into the actuators, while avoiding high ground impacts. The results of the take-off and landing optimization, as well as our actuation and power models, were successfully validated both in simulation and on our quadruped leg prototype, showing the effectiveness of our approach in maximizing hardware performance and the usefulness of exploiting combined impact-energy considerations during landing phases.

One of the main limitations of our approach is its computational complexity, which currently does not allow for online re-planning. Future work will involve the extension of our pipeline and, specifically, of the combined energy recuperation - impact mitigation concept to a full-fledged legged robot, possibly in an online fashion. This will require improvements of the current formulation and possibly the use of more efficient solvers.  
%The knowledge gained from our case study and our experimental trials will serve as a basis for the future development of the control and hardware of a new heavy-duty and agile electrically powered quadruped robot. 
% Given the limited capabilities of our setup, an interesting line for future research would be to extend and apply the core ideas of our formulations to full scale legged platform.

All the code employed to solve the optimizations, the  data, as well as the scripts used to produce the plots are made publicly available at~\cite{url::awesome_leg_repo}.