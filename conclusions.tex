\section{Conclusions}\label{sec:conclusions}
In this work we presented a case study built around the generation and execution of agile jumping maneuvers on a quadruped leg prototype.  
% In order to fully exploit and asses the performance of our single leg prototype, we employed accurate models of both the actuators and of the energy flow over the power bus. 
Empowered with accurate actuation and energy flow models, we explored the peculiar challenged associated to each of the phases of the jumping motion and proposed a two-stage solution to the generation of agile jumping maneuvers which relies on two separate TO problems: one for addressing the thrust phase with the objective of maximizing the jump height and the other to address the braking phase, with the objective of maximizing the energy regenerated into the actuators, while avoiding high ground impacts.
The results of the take-off and braking optimization were validated both in simulation and on the real hardware.
The knowledge gained from our case study and our experimental trials will serve as a basis for the future development of the control and hardware of a new heavy-duty and agile electrically powered quadruped robot. 
% Given the limited capabilities of our setup, an interesting line for future research would be to extend and apply the core ideas of our formulations to full scale legged platform.

All the code employed to solve the optimizations and the generated data, as well as the scripts used to produce the plots are made publicly available at~\cite{url::awesome_leg_repo} and~\cite{url::data_link}.